\documentclass[aspectratio=169]{beamer}
\usepackage[utf8]{inputenc}
\usepackage[portuguese]{babel}
\usepackage{graphicx}
\usepackage{listings}
\usepackage{xcolor}
\usepackage{tikz}
\usetikzlibrary{shapes,arrows,positioning}

% Tema
\usetheme{Madrid}
\usecolortheme{default}

% Configuração de código
\lstset{
    basicstyle=\ttfamily\footnotesize,
    keywordstyle=\color{blue},
    commentstyle=\color{gray},
    stringstyle=\color{red},
    showstringspaces=false,
    breaklines=true,
    frame=single,
    numbers=left,
    numberstyle=\tiny\color{gray}
}

% Informações do título
\title{Simulador do Algoritmo de Tomasulo}
\subtitle{Arquitetura Superescalar com Execução Fora de Ordem}
\author{Andre Luis, Caio Faria, Giuseppe Cordeiro, Vinicius Miranda }
\institute{PUC Minas}
\date{\today}

\begin{document}

% Slide 1: Título
\frame{\titlepage}

% Slide 2: Agenda
\begin{frame}{Agenda}
    \tableofcontents
\end{frame}

% Seção 1: Introdução
\section{Introdução}

\begin{frame}{O que é o Algoritmo de Tomasulo?}
    \begin{columns}
        \begin{column}{0.5\textwidth}
            \textbf{Características Principais:}
            \begin{itemize}
                \item Execução fora de ordem (out-of-order)
                \item Eliminação de hazards de dados
                \item Register renaming dinâmico
                \item Despacho dinâmico de instruções
            \end{itemize}
        \end{column}
        \begin{column}{0.5\textwidth}
            \textbf{Motivação:}
            \begin{itemize}
                \item Aumentar IPC (Instructions Per Cycle)
                \item Reduzir stalls por dependências
                \item Aproveitar paralelismo em nível de instrução
            \end{itemize}
        \end{column}
    \end{columns}
    
    \vspace{0.5cm}
    \begin{block}{Objetivo do Simulador}
        Ferramenta didática para auxiliar no entendimento da arquitetura superescalar e execução out-of-order
    \end{block}
\end{frame}

% Seção 2: Arquitetura
\section{Arquitetura do Simulador}

\begin{frame}{Componentes Principais}
    \begin{center}
    \begin{tikzpicture}[
        node distance=1.5cm,
        block/.style={rectangle, draw, fill=blue!20, text width=3cm, text centered, rounded corners, minimum height=1cm},
        arrow/.style={thick,->,>=stealth}
    ]
        \node [block] (iq) {Instruction Queue};
        \node [block, below of=iq] (rs) {Reservation Stations};
        \node [block, left of=rs, xshift=-2cm] (rob) {Reorder Buffer (ROB)};
        \node [block, right of=rs, xshift=2cm] (fu) {Functional Units};
        \node [block, below of=rs] (cdb) {Common Data Bus};
        \node [block, below of=rob] (reg) {Register File};
        
        \draw [arrow] (iq) -- (rs);
        \draw [arrow] (iq) -- (rob);
        \draw [arrow] (rs) -- (fu);
        \draw [arrow] (fu) -- (cdb);
        \draw [arrow] (cdb) -- (rob);
        \draw [arrow] (cdb) -- (reg);
        \draw [arrow] (rob) -- (reg);
    \end{tikzpicture}
    \end{center}
\end{frame}

\begin{frame}{Estruturas de Dados}
    \begin{columns}
        \begin{column}{0.5\textwidth}
            \textbf{Reservation Stations:}
            \begin{itemize}
                \item Armazenam instruções aguardando operandos
                \item Monitoram o Common Data Bus
                \item Despacham quando prontas
            \end{itemize}
            
            \vspace{0.3cm}
            \textbf{Reorder Buffer (ROB):}
            \begin{itemize}
                \item Mantém ordem do programa
                \item Permite execução especulativa
                \item Commit in-order
            \end{itemize}
        \end{column}
        \begin{column}{0.5\textwidth}
            \textbf{Common Data Bus:}
            \begin{itemize}
                \item Broadcast de resultados
                \item Forwarding automático
                \item Resolve dependências
            \end{itemize}
            
            \vspace{0.3cm}
            \textbf{Branch Predictor:}
            \begin{itemize}
                \item Predição de desvios
                \item Execução especulativa
                \item Recovery em misprediction
            \end{itemize}
        \end{column}
    \end{columns}
\end{frame}

% Seção 3: Implementação
\section{Implementação}

\begin{frame}[fragile]{Modularização do Código}
    \textbf{Estrutura do Projeto (Python):}
    \begin{lstlisting}[language=bash]
src/
├── core/
│   ├── simulator.py      # Lógica principal
│   └── structures.py     # RS, ROB, FU
├── mips/
│   └── parser.py         # Parser MIPS
└── gui/
    └── main_window.py    # Interface PyQt5
    \end{lstlisting}
    
    \vspace{0.3cm}
    \textbf{Principais Classes:}
    \begin{itemize}
        \item \texttt{TomasuloSimulator}: Coordena a simulação
        \item \texttt{ReservationStation}: Gerencia filas de instruções
        \item \texttt{ReorderBuffer}: Mantém ordem do programa
        \item \texttt{FunctionalUnit}: Executa operações
    \end{itemize}
\end{frame}

\begin{frame}{Pipeline de Execução}
    \begin{enumerate}
        \item \textbf{Issue (Despacho):}
        \begin{itemize}
            \item Aloca entrada no ROB
            \item Verifica disponibilidade de Reservation Station
            \item Register renaming via ROB tags
        \end{itemize}
        
        \item \textbf{Execute (Execução):}
        \begin{itemize}
            \item Aguarda operandos disponíveis
            \item Executa na Functional Unit apropriada
            \item Latências configuráveis por operação
        \end{itemize}
        
        \item \textbf{Write Result (Escrita):}
        \begin{itemize}
            \item Broadcast no Common Data Bus
            \item Atualiza ROB com resultado
        \end{itemize}
        
        \item \textbf{Commit:}
        \begin{itemize}
            \item Commit in-order via ROB
            \item Atualiza registradores e memória
            \item Libera recursos
        \end{itemize}
    \end{enumerate}
\end{frame}

% Seção 4: Interface Gráfica
\section{Interface Gráfica}

\begin{frame}{Funcionalidades da GUI}
    \begin{columns}
        \begin{column}{0.5\textwidth}
            \textbf{Visualização em Tempo Real:}
            \begin{itemize}
                \item Estado das Reservation Stations
                \item Conteúdo do Reorder Buffer
                \item Registradores e memória
                \item Common Data Bus
            \end{itemize}
        \end{column}
        \begin{column}{0.5\textwidth}
            \textbf{Controles:}
            \begin{itemize}
                \item Execução passo a passo
                \item Execução contínua
                \item Reset da simulação
                \item Carregar programas MIPS
            \end{itemize}
        \end{column}
    \end{columns}
    
    \vspace{0.5cm}
    \begin{block}{Objetivo Didático}
        Permitir que o estudante visualize cada etapa do algoritmo e entenda como as instruções fluem pelo pipeline
    \end{block}
\end{frame}

\begin{frame}{Execução Passo a Passo}
    \textbf{Benefícios Educacionais:}
    \begin{itemize}
        \item Ver instruções em diferentes estágios simultaneamente
        \item Observar resolução de dependências de dados
        \item Entender o funcionamento do register renaming
        \item Acompanhar execução especulativa e recovery
        \item Identificar ciclos de bolha e stalls
    \end{itemize}
    
    \vspace{0.3cm}
    \textbf{Highlight de Eventos:}
    \begin{itemize}
        \item Despacho de instruções
        \item Início e fim de execução
        \item Broadcast no CDB
        \item Commits
        \item Branch mispredictions
    \end{itemize}
\end{frame}

% Seção 5: Métricas
\section{Métricas de Desempenho}

\begin{frame}{Métricas Implementadas}
    \begin{columns}
        \begin{column}{0.5\textwidth}
            \textbf{Métricas Básicas:}
            \begin{itemize}
                \item \textbf{IPC}: Instructions Per Cycle
                \item \textbf{Total de Ciclos}: Duração da execução
                \item \textbf{Instruções Executadas}: Total
                \item \textbf{Ciclos de Bolha}: Stalls
            \end{itemize}
        \end{column}
        \begin{column}{0.5\textwidth}
            \textbf{Métricas Avançadas:}
            \begin{itemize}
                \item \textbf{Branch Mispredictions}: Taxa de erro
                \item \textbf{Utilização de FU}: Ocupação
                \item \textbf{ROB Stalls}: Bloqueios
                \item \textbf{Hazards Resolvidos}: RAW, WAR, WAW
            \end{itemize}
        \end{column}
    \end{columns}
    
    \vspace{0.5cm}
    \begin{block}{Fórmula do IPC}
        $$IPC = \frac{\text{Instruções Executadas}}{\text{Total de Ciclos}}$$
    \end{block}
\end{frame}

\begin{frame}{Análise de Performance}
    \textbf{Fatores que Afetam o IPC:}
    \begin{enumerate}
        \item \textbf{Dependências de Dados:}
        \begin{itemize}
            \item RAW (Read After Write) - verdadeira
            \item WAR (Write After Read) - resolvida por renaming
            \item WAW (Write After Write) - resolvida por renaming
        \end{itemize}
        
        \item \textbf{Hazards Estruturais:}
        \begin{itemize}
            \item Número limitado de Reservation Stations
            \item Número limitado de Functional Units
            \item Tamanho do Reorder Buffer
        \end{itemize}
        
        \item \textbf{Desvios Condicionais:}
        \begin{itemize}
            \item Precisão do preditor de desvios
            \item Custo de recovery em misprediction
        \end{itemize}
    \end{enumerate}
\end{frame}

% Seção 6: Demonstração
\section{Demonstração}

\begin{frame}[fragile]{Exemplo de Código MIPS}
    \begin{lstlisting}[language={[mips]Assembler}]
# Loop simples com dependências
ADDI R1, R0, 10    # R1 = 10 (contador)
ADDI R2, R0, 0     # R2 = 0 (soma)

LOOP:
    ADD R2, R2, R1     # soma += contador
    SUBI R1, R1, 1     # contador--
    BNEZ R1, LOOP      # if contador != 0 goto LOOP
    
# Resultado em R2 = 55 (soma de 1 a 10)
    \end{lstlisting}
    
    \vspace{0.3cm}
    \textbf{Observações:}
    \begin{itemize}
        \item Dependência RAW entre instruções do loop
        \item Predição de desvio para trás (taken)
        \item Paralelismo limitado devido às dependências
    \end{itemize}
\end{frame}

\begin{frame}{Execução no Simulador}
    \textbf{Passos da Demonstração:}
    \begin{enumerate}
        \item Carregar o programa MIPS
        \item Configurar parâmetros (RS, ROB, latências)
        \item Executar passo a passo para ver:
        \begin{itemize}
            \item Despacho de múltiplas instruções por ciclo
            \item Execução paralela quando possível
            \item Resolução de dependências via forwarding
            \item Funcionamento do preditor de desvios
            \item Commits in-order pelo ROB
        \end{itemize}
        \item Analisar métricas finais
    \end{enumerate}
    
    \vspace{0.3cm}
    \centering
    \textbf{Demonstrando a aplicação}
\end{frame}

% Seção 7: Conclusão
\section{Conclusão}

\begin{frame}{Resultados Alcançados}
    \textbf{Funcionalidades Implementadas:}
    \begin{itemize}
        \item[$\checkmark$] Simulador completo do Algoritmo de Tomasulo
        \item[$\checkmark$] Suporte a instruções MIPS (ALU, Load/Store, Branch)
        \item[$\checkmark$] Reorder Buffer para execução especulativa
        \item[$\checkmark$] Preditor de desvios (2-bit counter)
        \item[$\checkmark$] Interface gráfica educacional (PyQt5)
        \item[$\checkmark$] Execução passo a passo
        \item[$\checkmark$] Métricas de desempenho detalhadas
        \item[$\checkmark$] Exemplos didáticos de programas
    \end{itemize}
    
    \vspace{0.3cm}
    \begin{block}{Código Aberto}
        Disponível em: \texttt{github.com/giusfds/Tomasulo-Algorithm}
    \end{block}
\end{frame}

\begin{frame}{Trabalhos Futuros}
    \textbf{Melhorias Possíveis:}
    \begin{itemize}
        \item Implementar mais instruções MIPS (floating-point, etc.)
        \item Adicionar preditor de desvios mais sofisticado (gshare, tournament)
        \item Suporte a múltiplos níveis de cache
        \item Visualização 3D do pipeline
        \item Modo de comparação entre configurações
        \item Export de métricas para análise estatística
        \item Suporte a outras ISAs (ARM, RISC-V)
    \end{itemize}
\end{frame}

\begin{frame}{Conclusão}
    \begin{center}
        \Huge Obrigado!
        
        \vspace{1cm}
        
        \normalsize
        
        \textbf{GitHub:} github.com/giusfds/Tomasulo-Algorithm
    \end{center}
\end{frame}

\end{document}
